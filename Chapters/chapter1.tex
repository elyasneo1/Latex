\chapter{آموزش ورژن کنترل}




\section{مقدمه}
ورژن کنترل چیست ؟ و چرا باید به آن اهمیت دهیم ؟ ورژن کنترل یک سیستم ذخیره سازی یک یا چند فایل در طول زمان است \cite{Chacon2014}
ورژن کنترل سیستمی است که به توسعه‌دهندگان نرم‌افزار کمک می‌کند تا علاوه بر امکان مشارکت روی پروژه‌های نرم‌افزاری، بتوانند به تاریخچه‌ای از کدهایی که قبلاً نوشته‌اند نیز دست پیدا کنند و به طور کلی اهداف استفاده از سیستم‌های ورژن کنترل را می‌توان در موارد زیر خلاصه نمود:
- فراهم آوردن فرصتی برای توسعه‌دهندگان به منظور کار کردن به صورت هم‌زمان 
- مجزاسازی نسخه‌های توسعه داده شده اختصاصی تک‌تک توسعه‌دهندگان 
- نگهداری تاریخچه‌ای از هر نسخه از هر چیزی که به اشتراک گذاشته شود.
با استفاده از ورژن کنترل شما می توانید ایده های جدید خود را بدون نگرانی آزمایش کنید و در صورت نیاز به ورژن های قبلی برگردید[2] . ورژن کنترل سیستمی ضروری برای کار گروهی بر روی یک پروژه نرم افزاری است [3] . 
گیت یک نرم‌افزار کنترل نسخه و از مدل نرم‌افزارهای متن‌باز برای بازنگری کدمنبع توزیع شده و مدیریت منبع کد است که برای دنبال کردن تغییر فایلهای کامپیوتری و دنبال کردن کارهای انجام شده روی آن‌ها توسط افراد مختلف است. که در تمامی سیستم عامل های اصلی توسعه داده شده است [4] . گیت یک راه قدرتمند برای ردیابی و مقایسه نسخه ها، رفع خطاها، کشف رویکردهای جدید به شیوه ای ساختاری است[5] .
گیت ابتدا برای توسعه لینوکس توسط لینوس تُروالدز به وجود آمد و اکنون پروژه‌های فراوانی از آن الهام گرفته‌اند. هر دایرکتوری کاری در گیت یک مخزن کامل با تاریخچه کامل تغییرها و قابلیت بازنگری آن‌ها است و برای کار با آن نیازی به دسترسی به شبکه یا سرور مرکزی وجود ندارد.
امروزه برنامه نویسی که به گیت مسلط نباشد را عملا برنامه نویس نمیدادند، به عبارت دیگر تسلط به گیت وظیفه ی هر برنامه نویس میباشد. در هرپروژه ای در هر سطحی در دنیا و در هر شرکتی از گیت استفاده میشود حتی شرکت های غیر برنامه نویسی مانند مقاله نویسی ، پایان نامه و هر نوع فایل متنی دیگری میتوان از گیت استفاده نمود؛ پس با ما در ادامه این آموزش همراه باشید چون با همین اطلاعات و تمرین آن ها میتوانید آشنایی با گیت را در رزومه خود اضافه کنید.

\section{تاریخچه}
اینجا ی چیزایی راجبه تاریخش بنویس و سع یکن رفرنس بدی بجز اینجا جای دیگ رفرنس نداریم 
\section{کنترل منابع \lr{(Source control)}}
ی چیز در مورد سرس کنترل بنویس بازار و cvn و ..... اینا رو هم بگو 



\section{گیت \lr{(Git)}}
ی چیزایی در مورد گیت و ی اشارات کلی به sub sction ها بکن 
\subsection{شروع کار با گیت}
ی چیزی کلی راجبه init و clone 
\subsubsection{init}
راجبه init 
\subsubsection{clone}
راجبه clone
\subsection{بررسی و دستیابی}
ی چیز کلی راجبه status و checkout 
\subsubsection{status}
راجبه status 
\subsubsection{checkout}